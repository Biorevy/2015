\documentclass[a4paper,11pt]{article}

\usepackage{revy}
\usepackage[utf8]{inputenc}
\usepackage[T1]{fontenc}
\usepackage[danish]{babel}

\revyname{Biorevy}
\revyyear{2015}
\version{0.2}
\eta{$3:18$ minutter}
\status{Ikke færdig}
\responsible{Thomas BT}

\title{Biologisk Korrekte Børnesange}
\author{Helene, Danny, Hygge, Valdbjørn}
\melody{Forskellige Børnesange}

\begin{document}
\maketitle

\begin{roles}
    \role{Bio}[Isabella] Syngende, biolog-pædagog
    \role{Børn}[Amanda] Barn
    \role{Børn}[Siren] Barn
    \role{Børn}[Jeppe] Barn
    \role{Krok}[Jeppe] Krokodille
\end{roles}

%\begin{props}
% RevyTeX kan ikke lide tomme environments.
%\end{props}

\scene biolog sidder på scenen -- Børnene dissikerer

\scene biolog stopper dissektionen, og starter sanglegen.

\begin{song}

\sings{Bio}Lille Petra edderkop
Sad så fint i spindet
Så kom fluen
klistrede sig fast
Så kom Petra 			
forgifted' fluens krop
Lille Petra edderkop
spiste så sin mand
\end{song}

\begin{sketch}

\says{Barn} Hvorfor spiste hun sin mand?!
\says{Bio} Mænd skal kun bruges til reproduktion -- og så kan man jo ligesågodt få et måltid ud af det også \ldots

\end{sketch}
\begin{song}
Tangokat
Min kat den fanger mus og leger med dem
Og den er ikke sulten
Den gør det kun for sjov
Min kat den er en psyko-patisk dræber
Og så den verdens sød’ste lille missekat

\end{song}
\begin{sketch}

\says{Børn} Jamen, hvad med musen?
\says{Bio} Den kan vi dissekere i morgen!

\end{sketch}
\begin{song}			
Den tykke flodhest er så ond
den løber med en rumlen,
og da jeg ikk var hurtig nok,
så endte jeg i vommen.
Oja-oja uhaha, oja-oja uhaha,
og da jeg ikk var hurtig nok,
så endte jeg i vommen.

\end{song}
\begin{sketch}

\says{Børn} Jamen, er flodheste ikke planteædere?
\says{Bio} Jo! Men ikke når der er irriterende børn!

\scene Krokodille lister ind, under sidste vers, når sangen er slut, overfalder den børnene

\end{sketch}

\begin{song}
Hvis du ser en krokodille i dit badekar
bør du virk’lig være bange, den er ikke rar
for(den)  kan spise både dig, din mor og din far
\end{song}



\end{document}

