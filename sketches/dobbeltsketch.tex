\documentclass[a4paper,11pt]{article}

\usepackage{revy}
\usepackage[utf8]{inputenc}
\usepackage[T1]{fontenc}
\usepackage[danish]{babel}

\revyname{Biorevy}
\revyyear{2015}
% HUSK AT OPDATERE VERSIONSNUMMER
\version{0.3}
\eta{$6:42$ minutter}
\status{Færdig}
\responsible{En revyst}

\title{Dobbeltsketch}
\author{Daniel, Stine \& Valdbjørn}

\begin{document}
\maketitle

\begin{roles}
    \role{S1}[Danny] Studerende 1 (i begge)
    \role{S2}[Tristan] Studerende 2 (i eksamen)
    \role{E}[Laurids] Eksamensvagt
    \role{V1}[Amanda] Sød vejleder
    \role{V2}[Valdbjørn] Fordrukken vejleder
    \role{R1}[Annette] Rus 
    \role{R2}[Miriam] Rus 
    \role{F}[Nathalia] Fremdriftsreformen 
    \role{AV}[AV] AV
\end{roles}

\begin{props}
    \prop{Bord}[Person, der skaffer] 1 stk. forkromet Biorevy$^{tm}-bord^{tm}$ (patent pending)
    \prop{Stole}[Person, der skaffer] 2 stk.
    \prop{Rushatte}[Genbrug fra 10 små biorus] Til tre russer.
    \prop{Skilte}[Person, der skaffer] Med slogan?
    \prop{Eksamensopgaver}[Person, der skaffer]
    \prop{Eksamensting}[Person, der skaffer]
    \prop{Tyggegummi}[Person, der skaffer]
    \prop{Øl, alkohol}[Person, der skaffer]
    \prop{Beerpongkopper}[Person, der skaffer]
\end{props}


\begin{sketch}


\scene Sketch A spilles alene på scenen. fremdriftsreformen siger at den er dumpet, og skal tages igen. 
Næste gang de går på, skal den spilles samtidig med sketch B (fordi fag ikke kan udskydes under fremdrift). 
Det er her magien sker, for nu smelter to sketches sammen til en, med forvirring og interaktioner mellem de to sketches. 

\scene{Lys op.}

\scene A
To studerende sidder ved bord på scenen med deres eksamensnoter, stillet vand frem, og gør klar til skriftlig eksamen.
Eksamensvagten kommer ind. Skal være meget overspillet, næsten nazi agtig, med hårde skridt og kraftige bevægelser.
Han ånder dem i nakken og kikker over skulderen, etc.
Eksamen går i gang:

\says{E} Nu starter eksamen i molekylærbiologi, I har tre timer i absolut stilhed.

\act{Studerende kigger på hinanden, med panik i øjnene}
\act{S1 og 2 rækker hånden op}

\says{S1} hvorfor må vi ikke tisse?


\says{E} Hvis I kan flyve til New York uden at ryge, kan I sagtens sidde 4 timer uden at tisse. Ikke flere spørgsmål, vi starter nu

\says{S} \act{tager hånden ned igen}

\scene Eksamensvagt deler opgaver ud, og de går I gang. Det er tydeligvis svært, der bliver svedt over opgaverne.
Eksamensvagten går rundt og er nazi

S gaber. Eksamensvagten kigger ondt på dem.

\says{E} I SKAL VÆRE STILLE

\says{S1} Jeg kan godt høre dig, jeg sidder lige her

\says{E} I SKAL VÆRE STILLE!

\says{S1} \act{tygger tyggegummi, deler ud til den anden}

\scene Eksamensvagten ser ikke at der bliver byttet ting, men reagerer på at de tygger, og siger igen at de skal være stille

\says{E} JEG KAN GODT SE I SNAKKER SAMMEN; I MÅ IKKE SNAKKE OM OPGAVEN

\scene Han er godt nok nazi siger de mens han vender ryggen til – E reagerer ikke

\act{S1 ser fortvivlet ud, har glemt sin replik}

\says{S2}. Han er godt nok nazi

\act{S1 har \emph{stadig} glemt sin replik}

\says{E} Ej, helt ærligt, har du glemt din replik igen?!

\scene Enter Fremdriftsreformen
\says{S2}[opgivende] Ej, Fremdriftreformen igen, hun skal da også bare blande sig I alting

\says{F} Hallo! I må ALDRIG bryde character. I dumper den her sketch. I har VÆRSGO at tage den igen med det samme!

\says{S2} Men der går jo en anden sketch på lige om lidt?

\says{S1} Ja, og jeg har en rolle I næste sketch?!

\says{F} Det er bare ærgerligt, så måtte du have læst mere på manus. Ud med jer, det er tid til sceneskift.

\scene Panikken lyser ud af dem.

\scene Lys ned.

\scene \textbf{Herfra kører to sketches i parallel}; eksamenssketchen starter forfra, og kører parallelt med en ny sketch

\scene Lys op. Måske er E ikke nået ud endnu, fordi det har været en meget kort mørkepause.

Studerende sidder ved bordet. Eksamensvagten kommer ind.

\act{S1 og S2 tager deres tyggegummi ud og sætter dem på bordet.}

\says{E} Nu starter eksamen i molekylærbiologi, I har tre timer i absolut stilhed.

\scene Studerende kigger på hinanden, med panik i øjnene

\act{S1 og 2 rækker hånden op}

\says{E} Nu starter eksamen i molekylærbiologi, I har tre timer i absolut stilhed.

\says{S1} Hvorfor må vi ikke tisse?

\says{E} Hvis I kan flyve til New York uden at ryge, kan I sagtens sidde 4 timer uden at tisse. Ikke flere spørgsmål, vi starter nu.

\act{S1 \& S2 tager hånden ned igen}

\scene Idet eksamensvagten skal til at dele opgaverne ud, går det op for dem at opgaverne stadig ligger på bordet. De giver dem febrilsk tilbage, for derefter at få dem udleveret igen.

\scene Eksamensvagt deler opgaver ud, og de går I gang. Det er tydeligvis svært, der bliver svedt over opgaverne. Eksamensvagten går rundt og er nazi.

\scene Et rushold med vejledere kommer ind. Alle russerne har rushatte på.
Der bliver kastet en kasket over til S1, S1 tager hatten på, rejser sig op og deltager I rusturen.

\says{V1} Vi har lavet det meget inkluderende slogan i Rusvejledningen i år, for at vise at der ikke er nogen der skal føle sig presset, og at her er plads til alle: Sloganet er: Drik (med) Respekt, Ingen Gruppepres. Det er lidt langt, så vi forkorter det mundret til DRIG.

\says{AV} Slide: \texttt{``Drik (med) Respekt, Ingen Gruppepres (DRIG)''}

\says{V1} Lad os lige sige det lidt i kor: DRIG, DRIG, DRIG, DRIG

\says{V2} Jaja, men selvom vi ikke må \emph{LÆGGE} op til druk, vil jeg gerne \emph{DÆKKE} op til druk. Så kan I jo i princippet vælge om I vil være med til at drikke eller ej\ldots

\scene De begynder at line øl og alkohol etc. op på eksamensbordet. Stiller op til beer pong

\scene S1 \& S2 gaber. Eksamensvagten kigger ondt på dem.

\says{E} I SKAL VÆRE STILLE

\says{S2} Jeg kan godt høre dig, jeg sidder lige her

\says{E} I SKAL VÆRE STILLE!

\scene Der bliver spillet beer pong på eksamensbordet, og Eksamensvagten og S2 får bolde i hovedet. (Det er en anelse forstyrrende for dem). S1 er med i spillet, men står tæt på den stol hun sidder på. Hun sætter sig ned.

\act{S2 tygger tyggegummi, deler ud til den anden}

\scene S2 tilbyder S1 det tyggegummi der sidder på bordet, tager imod det, tygger videre.

\scene Eksamensvagten ser ikke at der bliver byttet ting, men reagerer på at de tygger, og siger igen at de skal være stille

\says{E} JEG KAN GODT SE I SNAKKER SAMMEN; I MÅ IKKE SNAKKE OM OPGAVEN

\scene Det var blevet S1's tur i beer pong, så hun er nødt til at rejse sig op fra bordet, er lige ved at glemme rushatten\ldots De andre russer råber ret højt da hun ikke rammer, og larmer, hujer etc.

\says{S2} Han er godt nok nazi \act{vender ryggen til – E reagerer ikke}

\act{S1 står og føler sig råbt ind i hovedet -- }
\says{S1} Ej, kan I ikke være lidt stille, det er jo lige til at blive \ldots  døv af\ldots \act{et lys går op for hende}

\act{S1 smider rushatten, kaster sig ned på stolen}

\says{S1} Så kan man la' vær' med at tage et kursus, hvor Hitler er censor!

\act{S1 er meget stolt over at kunne huske sin replik!}

\scene Fremdriftsreformen kommer ind igen, klappende.

\says{F} Meget fornemt, det er acceptabelt. Sketch A, I er bestået. Russketch: hvor er jeres punchline?

\says{V1} Vi har haft så mange andre krav at vi ikke har nået at skrive nogen

\says{F} Det kan jeg ikke tage mig af – det må være jeres eget ansvar. Vi ses til næste år.

\scene Vejlederne bliver meget skuffede... Særligt S1 er depri.

\says{S1} Men jeg har jo slet ikke haft tid, med min anden resketch?!!!!!!! FUCK DET HER LORT, JEG DROPPER UD AF REVYEN!

\says{F} Sublim effektivisering.

\says{E} \ldots og så syntes I at \emph{jeg} var streng\ldots

\scene{Lys ned.}

\end{sketch}
\end{document}
