\documentclass[a4paper,11pt]{article}

\usepackage{revy}
\usepackage[utf8]{inputenc}
\usepackage[T1]{fontenc}
\usepackage[danish]{babel}

\revyname{Biorevy}
\revyyear{2015}
% HUSK AT OPDATERE VERSIONSNUMMER
\version{0.3}
\eta{$4:04$ minutter}
\status{Næsten færdig}
\responsible{\texttt{tex@biorevy.dk}}

\title{Eksaminerings-ception}
\author{Nathalia}

\begin{document}
\maketitle

\begin{roles}
    \role{Stu}[Annette] Studerende
    \role{Eks}[Tristan] Eksaminator
    \role{KU}[Amanda] Repræsentant fra KU
    \role{Cen}[Ditte] Censor for den studerende
    \role{Min}[Siren] Ministeriel repræsentant
    \role{Spe}[NN] Intro speaker
    \role{AV}[AV] AV -- Elevator-ventemuzak
\end{roles}

\begin{props}
    \prop{Stole}[Person, der skaffer] Evt stole? Dog kan det give et mere dynamisk udtryk hvis alle står op. Studerende skal i hvert fald stå op (mundtlig eksamen)
    \prop{PowerPoint-slide}[Person, der skaffer] Et enkelt slide fra et powerpoint kan være en nice idé men ikke nødvendigt
    \prop{Parpir \& pen}[Person, der skaffer] 3 x blok og pen (til Eks, Rep og Cen)
    \prop{Jakkesæt}[Person, der skaffer] Jakkesæt til repræsentanten (Han er fra Finans-afdelingen)
\end{props}


\begin{sketch}

\scene{Beskrivelse}

\says{Spe}Københavns Universitet er igennem den seneste tid blevet truet af ministeriet med færre midler, hvis ikke studietiden nedsættes. Derfor er KU begyndt til hver eksamen at udsende en repræsentant fra økonomi afdelingen for at vurdere undervisernes præstationer, og få flere studerende igennem eksaminerne i første forsøg.

\scene Lys op. På scenen er en mundtlig præsentation ved at blive afsluttet, studerende står og præsenterer for eksaminator og censor, KU rep står i hjørnet.

\says{Eks}Meget flot præsentation. Kan du fortælle os om Staphylococcus aureus er grampositiv eller 
gramnegativ?

\says{Stu}Ja, den er gramnegativ.

\says{Eks}Ja, nej det er desværre ikke helt rigtigt, men godt bud!

\act{Cen hiver Stu's papirer over og læser i dem}

\says{KU}\act {Ryster let på hovedet og skriver ting ned på sin blok.}

\says{Cen}Kan du fortælle os lidt om gramfarvning og hvad det kan gøre?

\says{Stu}Altså gramFARVNING\ldots Hmm, det farver jo cellerne og så\ldots Hmm\ldots \act{Hiver sine noter tilbage, og kigger i dem} Kan man se hvor hurtigt de svømmer?

\says{Eks}\act{meget skuffet} Mjaa, næsten rigtigt\ldots Det er jo for at skelne mellem grampositive og gramnegative.. Men lad os nu lade det ligge! \act {Kigger meget nervøst hen på KU som skriver en masse noter ned.}

\says{KU}[hvisker for sig selv] Det er simpelthen også ALT for svære spørgsmål de stiller her, det har man jo INGEN chance for at vide!

\says{Eks}[lettere panisk] Øhm okay! Et sidste spørgsmål så inden vi slutter. \act{kigger i sine noter og smiler glad, da det er et let spørgsmål} Hvordan kan en celle bevæge sig?

\says{Stu}[Hurtigt, nærmest udråbende] Med sine ben?!

\says{Eks}\ldots Øhm, næsten rigtigt, det er mere en slags arme, kan du huske hvad de hedder?

\says{KU}[ryster på hovedet, sukker, trækker på skuldrene] Altså, altså\ldots

\says{Stu}Øhm\ldots Tentakler?

\says{Eks} Næææj, ikke heelt, her er der kun en, den sidder sådan for enden\ldots \act{peger på tegningen på overhead/præsentation}

\says{Stu}Øhm, haler?

\says{Eks} Tættere på\ldots mere sådan \act{illustrerer flagel over sit eget hovede}

\act{Cen er utilfreds med at Eks hjælper så meget}

\says{Stu} En propel?!

\act{Eks sukker, har givet op}

\scene KU har kigget forvirret med, kigger på Eks's noter, ryster på hovedet, skribler som en gal

\says{Cen}Nå, men, SÅ er tiden vidst også gået.

\says{Eks}Nåh ja, vil du gå ud og så snakker censor og jeg lige?

\scene {Stu går ``ud'' (ovre på den anden side af scenen)}

\says{Eks}Det gik jo fint, ikke? Klart 4-tal!

\says{C} For den præstation?! Har du spist søm?!

\says{Eks} Kom nu -- så bare et 2-tal? \ldots pleease?

\says{C} Nej.

\says{Eks} Men -- hvis hun dumper, så dumper jeg også!

\says{KU}[Lidt strengt] Som repræsentant fra KU må jeg sige, at jeg er meget skuffet over din Studietidsnedsættelsesindsats. Lad os håbe for dine forskningsmidler, at det går bedre med de næste studerende, ellers bliver vi nødt til at evaluere din indsats

\says{Eks} Men, men, jeg er faktisk god til at undervise!

\says{KU} Du gør tydeligvis ikke nok for at hjælpe KU med at nedsætte studietiden.

\says{Eks} Men, det var Censoren fra Aarhus der gjorde det!

\says{KU} Hvis jeg skal tro på det bliver jeg nødt til at se nogle resultater! 

\scene Eks forlader scenen, helt nedslået

\says{Min} Som repræsentant for ministeriet må jeg sige at at jeg er yderst skuffet over jeres studienedsættelsesindsats!

\says{KU} Du mener vel StudieTIDSnedsættelse?

\says{Min} Jaja\ldots Studie-``tids''-nedsættelse, lad os sige det. 

\says{KU} Jojo, men det bliver bedre! Tro mig, det bliver meget bedre -- jeg er blevet lovet at de næste studerende klarer det bedre\ldots

\says{Min} Hvis jeg skal tro på det, så skal jeg fandme se nogle resultater! I bliver nødt til at tilpasse det faglige niveau. Hvis I ikke kan få de studerendes niveau OP, må I få det faglige niveau NED. 

\says{KU} Men hvis de studerende dumper eksamen, så mister vi vores midler og så falder niveauet helt af sig selv. Så er mit arbejde jo gjort\ldots Så har jeg jo tid til kaffe.

\scene Cen og KU forlader scenen. Eks står tilbage, stirrer tomt, deprimeret, opgivende ud på publikum.



\scene{Lys ned.}

\end{sketch}
\end{document}
