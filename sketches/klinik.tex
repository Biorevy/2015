\documentclass[a4paper,11pt]{article}

\usepackage{revy}
\usepackage[utf8]{inputenc}
\usepackage[T1]{fontenc}
\usepackage[danish]{babel}

\revyname{Biorevy}
\revyyear{2015}
% HUSK AT OPDATERE VERSIONSNUMMER
\version{0.2}
\eta{$4:44$ minutter}
\status{Næsten Færdig}
\responsible{Thomas BT}

\title{Klinik for Kontrolleret Selektion}
\author{Allie}

\begin{document}
\maketitle

\begin{roles}
    \role{Dok1}[Siren] Doktor (M/K)
    \role{Dok2}[Laurids] Doktor (M)
    \role{Kars}[Tristan] Karsten
    \role{LiLo}[Isabella] Liselotte
\end{roles}

\begin{props}
    \prop{Bord}[Person, der skaffer] Et Bord -- hidtil uset i Biorevy$^{(TM)}$
    \prop{Stol}[Person, der skaffer] En stol
    \prop{Stol}[Person, der skaffer] En stol
    \prop{Stol}[Person, der skaffer] En stol
    \prop{Papier}[Person, der skaffer] Papirer, der kan ligge på Bordet
\end{props}


\begin{sketch}


\scene Lys op

\scene Dok 1 sidder ved bordet. Dok2 går rundt og ordner papirer i baggrunden. Liselotte og Karsten kommer ind. De hilser på doktorerne og giver hånd.

\says{Kars} Goddag. Karsten

\says{LiLo} Liselotte

\says{Dok1} Goddag. Velkommen til Klinikken for Kontrolleret Selektion. Mit navn er Doktor Streg, og jeg skal stå for interview og bedømmelse i dag. I er selvfølgelig berettiget til en second opinion, i tilfælde af at jeres ansøgning om at få lov at reproducere ikke bliver godkendt.
 
\scene De bliver henvist til de to tomme stole ved bordet.  

\says{LiLo} Det var da dejligt, hva' Karsten?

\says{Kars}[gnavent] det er også så besværligt med alle de hersens nye regler. Bare for at få lov at få et barn.

\says{Dok1} Jaja, men tænk bare på hvordan det var før i tiden! Folk rendt bare rundt og fik børn på må og få.

\says{Dok2} Ja, fuldstændig uansvarligt

\says{Dok1} Og uden tanke på menneskehedens fremtid

\says{Dok2} Ulækkert!

\says{Dok1} Uanstændigt

\says{Dok2} Og derfor mine herskaber, har staten nu sat dejligt skik på hele affæren.

\says{Dok1} Ja, det har de nemlig. Efter den her nye lov, skal alle mænd og kvinder – som i ved – undersøges og godkendes før de får lov til at reproducere, så vi kan sikre os at kun de bedste gener går videre. Og den menneskelige race forbedres gennem kontrolleret avl.

\says{Dok2}[saligt]Det er smukt.

\says{LiLo} Ja, det er sandelig. Æh\ldots Meget dejligt. Ikke Karsten?

\says{Kars}[tøvende] Jooo\ldots Det er det vel.

\says{Dok1} Ja det er det! Og lad os så få kigget på det.

\act{Dok1 kigger i sine papirer.}
 
\says{Dok1} Ja, ja, ja, ja, ja, ja. Det ser jo helt fint ud. Jeg ser du både har en elitesvømmer og en doktorgrad i biokemi i dit slægtstræ Liselotte. Ja ja. Lidt mindre godt ser det ud hos dig, Karsten

\says{Kars} Der er tre doktorgrader i min familie!

\says{Dok1} Karsten, både du og jeg ved jo godt, at humanistiske doktorgrader ikke rigtigt tæller. Nå! Men jeg kan se at der også er sklerose, skaldethed og blodpropper i din familie, Karsten
\says{Kars} Skaldethed er vel dårligt en dårlig egenskab.  

\says{Dok2} Aaaarrrhhh. Det kommer vel an på hvordan man ser på det

\says{Dok1} Det gør det nemlig, og jeg er sikker på, at også du, Liselotte, vil være enig i, at en mand med hår er bedre end en mand uden. 

\act{De smiler sigende til hinanden og ser afventende på Liselotte.}

\says{LiLo} Jooeh\ldots Jo, altså jeg kan godt lide hår.

\says{Dok1} der ser De! Derfor kan vi desværre ikke imødekomme Deres ønske om at formere jer.

\says{Kars} Vi klager!

\says{LiLo} Jamen, vi ønsker os sådan et barn. Er der ikke andet vi kan gøre?

\says{Dok1} Jo, der er altid muligheden for fremavling af de egnede gener hos den ene part

\says{LiLo} Jeg tror ikke lige -

\says{Dok1} Jo, fru Liselotte, Deres gener er som sagt fremragende, så jeg vil foreslå dem, at avle med et mere, øh, skal vi sige passende, hanligt eksemplar.

\says{Kars} Nej, hør nu!

\says{Dok1} Lad mig se.. Hm, åh ja. jeg kan foreslå min kollega doktor Jonathan her! \act{peger på dok2, der straks gør sig til}. Han har fremragende gener. Sund krop, intelligent, stærke tænder, tæt kropsbehåring. Ja, se, det ville der komme gode børn ud af.

\says{LiLo} Jamen hvad så med Karsten?

\says{Dok1} Ja, vi blir nok nødt til at skyde ham.

\scene Dok1 skyder Karsten

\says{LiLo}. Karsten! Jamen dog! Jeg, jeg tror godt, jeg kunne tænke mig den der second opinion. 

\says{Dok1} Det skal De få lille De. Jonathan!

\says{Dok2} Ja, ja, lad mig se. (kigger papirerne igennem) Nej, der var ikke noget at gøre, er jeg bange for Liselotte. (Går hen til Liselotte og lægger en arm om hende) Deres mand var ikke værd at avle på. Og ærlig talt, hvad kan en mand bruges til, hvis han ikke kan avles på. 

\says{LiLo} Jamen hvad gør jeg nu?
\says{Dok2} Det skal jeg sige Dem. De er et fremragende eksemplar på den menneskelige race, og det er gener, der skal avles på. Så kom du bare med mig. 

Doktorne synger her evt en sang om at mennesker kke selv kan styre naturlig selektion og derfor skal have hjælp til at kontrollere deres avlen. 


\scene{Lys ned.}

\end{sketch}
\end{document}
