\documentclass[a4paper,11pt]{article}

\usepackage{revy}
\usepackage[utf8]{inputenc}
\usepackage[T1]{fontenc}
\usepackage[danish]{babel}

\revyname{Biorevy}
\revyyear{2015}
% HUSK AT OPDATERE VERSIONSNUMMER

\version{0.3}
\eta{$4:52$ minutter}
\status{Ikke færdig}
\responsible{\texttt{tex@biorevy.dk}}

\title{Miljøafviklingssvin}
\author{Danny og Stine}

\begin{document}
\maketitle

\begin{roles}
    \role{Eva}[Jennie] Eva Kjer Hansen
    \role{Sv}[Danny] Svineavler
    \role{D1}[Annette] Demonstrant 1
    \role{D2}[Allie] Demonstrant 2
    \role{D3}[Jean] Demonstrant 3
    \role{Spe}[NN] Speaker 
\end{roles}

\begin{props}
    \prop{Rekvisit}[Person, der skaffer] Telefon 
    \prop{Rekvisit}[Person, der skaffer] Høtyv
    \prop{Rekvisit}[Person, der skaffer] Demonstrantskilte
\end{props}


\begin{sketch}

\scene{En ud af tre Eva Kjer Hansen sketches.}
\scene{Lys op.}

\says{Spe}Nyudnævnte Miljø og Fødevareminister Eva Kjer Hansen er taget på rundtur til danske svineavlere for at få et nærmere indblik i deres udfordringer og hvordan regeringen kan hjælpe med at øge landbrugets konkurrenceevne.

\scene{Svineavler og Eva står på scenen}

\says{Sv} Goddaw, welkommen til. Kan jeg by’ på en brunsvi'r?

\says{Eva} Øhm, nej tak, en latte macchiatto kunne jeg dog godt bruge?

\says{Sv}\ldots Er det en drink?

\says{Eva}Nej, nej \ldots Men det er også lige meget. Den nye gode regerings grønne realisme, betyder jo at der skal være plads til landbruget i miljøhensynene. Så jeg er kommet for at høre , hvad regerngen kan gøre for DIG.

\says{Sv}Jo jo, da. Jeg har nogle spørgsmål.

\says{Eva}Kom med dem?

\says{Sv}Jo, hvordan med det der gylle? Sååå… Kan jeg bare sprede det ud på min mark som jeg vil?

\says{Eva} Ja! Vi har lige sat grænserne op for gyllespredning, så du går bare i gang!

\says{Sv}Joo, ja, det lyder da dejligt. Men øh, jeg vil helst ik' spørge, men som miljøminister er du nok uddannet i miljø, så må du ligesom vide lidt mere end mig\ldots

\says{Eva}Næ, jeg er cand.polit

\says{Sv}Nånåda\ldots Men som miljøminister må du vide noget om miljø\ldots Det er jo sådan, at jeg er begyndt at give smågrisene Zinc og kobber i foderet - er det så stadig fint at sprede på markerne?   

\says{Eva}Jajada, det' INTET problem. Tungmetaller har ALDRIG slået nogen ihjel. 

\says{Sv}Nå okay da, det lyder super. Jo, og så tænkt' jeg på de der randzoner der. 

\says{Eva}Ja, de er simpelthen også så besværlige. Hvad siger du til at vi sløjfer dem?

\says{Sv} Jo da, det ville jo immervæk gøre tingene lidt lettere\ldots Men går det så ikke ud over det dersens ``vandmiljø''?

\says{Eva} I don’t Care\ldots Vandløb skal jo alligevel bare lede vand, de er alligevel ret overvurderede \ldots

\says{Sv} Joeh\ldots

\says{Eva} Er der ikke også noget med at de leder hurtigere når de er lige? Det må vi også lige få rettet op på \ldots

\says{D1, D2, D3}\act {Demonstranter kommer ind. Har kampråb:\says{D1, D2, D3} EVA Kjer, ta' dog og la' vær'!}

\says{D1}Hvordan kan du sove om natten når du smadrer vores miljø?!

\says{Eva}Helt ærligt? Det går ret nemt for I really don’t care! Vi i Venstre lægger vægt på en Grøn Realisme, og det går lissom ud over vores konkurrenceevne hvis vi begrænser landbrugsarealet og gødskningen\ldots

\says{D2}[afbryder] Hvor er det grønne i jeres realisme?!!!

\says{Eva}Oppositionspartierne klandrer os for ikke at være grønne nok, men algeopblomstringer er jo meget grønnere end gedder så I don’t Kjer. Så længe…

\says{D3}\act{Prøver at afbryde}

\says{Eva}\act{Tysser på demonstranter} 
\says{Eva}Så længe EU ikke går foran med miljøkrav, vil vi ikke overimplementere miljøkrav. Det handler om grøn realisme.

\says{D3}Men hvad så med tungmetallerne?

\says{D3}\act{Henvendt til svineavleren} Hvordan bilder du dig ind at fodre grise med tungmetaller?

\says{Sv} Jojo, men ellers får de jo diarré, når jeg tager dem fra moderen. Og jeg vil jo gerne have et alternativ til antibiotika, som ikke giver sådan noget resistent i MRSA. Jeg vil gerne undgå det, men der er ligesom ikke noget alternativ.

\says{D1}Du kunne lade dem blive hos moderen?

\says{Sv}Nej nej nej nej nej nej nej nej\ldots Nej nej nej\ldots Jeg taber i forvejen 80 kr pr slagtesvin\ldots

\says{Eva} Jam'n, det skal jo også kunne betale sig at være landmand.

\act{Eva begynder at liste sig ud bag scenetæppet}

\says{D2}Har du overvejet at omlægge produktionen?

\says{Sv}Nej nej nej nej nej nej nej ej ej nej nej nej\act{Pause\ldots} nej nej nej nej\act{Pause igen\ldots} nej nej (evt i MIN familie har vi altid avl't grise)
\act{D3 ser pludselig, at Eva er væk! Prikker til de andre demonstranter}
\says{D2}Hey! Eva er smuttet -- hun forsøger at undvige!
\says{D1} What! Hun skal stilles til ansvar -- efter hende!

\scene Demonstranterne forlade landmanden på scenen. Han står måske stadig og siger ``Nej''

\scene Lys ned














\end{sketch}
\end{document}
